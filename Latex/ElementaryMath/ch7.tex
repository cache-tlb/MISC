\chapter{微积分}

\noindent 橡皮绳上的蚂蚁

有一条橡皮绳, 初始长度为 $L$, 有一只蚂蚁在左端点, 沿着绳子向右端点爬, 速度为 $ V_1 $. 橡皮绳左端点固定, 右端点以速度 $ V_2 $ 被拉长, 绳子各处是均匀拉长的, 并且可以拉到无限长. 问蚂蚁能否爬到右端点, 如果能, 需要多长时间? 一组具体的数值: $ L = 100, V_1=V_2=1 $.

\noindent 解:

设左端点为原点, 时间为 $ t $ 时, 绳子右端点位置为 $ L + V_2 t $, 蚂蚁的位置为 $ s $. 蚂蚁的绝对速度由它的爬行速度和绳子伸长的速度叠加而来, 绳子伸长贡献的速度取决于蚂蚁当前的位置, $ s $ 与 $ t $ 的关系为:
\[
\dv{s}{t} = V_1 + V_2\cdot\frac{s}{L+V_2 t}
\]

这是一个 $ y' + P(x)y = Q(x) $ 形式的常微分方程, 可以直接套公式求. 也可以做如下的代换: $ u = \dfrac{s}{L+V_2t} $, 则 $ s = (L+V_2t)u $, 这里 $ u\in[0,1] $ 表示当前蚂蚁位置相对于绳长的比例. 于是
\begin{align*}
s &= u(L+V_2t) \\
\dd{s} &= (L+V_2t)\dd{u} + V_2u\dd{t} \\
(L+V_2t)\dv{u}{t}+V_2u &=V_1+V_2u \\
\dv{u}{t} &=\frac{V_1}{L+V_2t} \\
u &= \frac{V_1}{V_2}\ln{(L+V_2t)} + C
\end{align*}

当 $ t = 0 $ 时, $ u = 0 $, 所以 $ C = -\dfrac{V_1}{V_2}\ln(L) $. 也就是 $ u = \dfrac{V_1}{V_2}\ln{(1+\dfrac{V_2}{L}t)}$ .

代入数值, 求出 $ u = 1 $ 时的解为 $ t = \left(e^{{V_2}/{V_1}}-1\right)\dfrac{L}{V_2} = 100(e-1)$.




\newpage

%------------------------------------------------------------------------------%


















